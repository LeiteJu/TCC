%!TeX root=../tese.tex
%("dica" para o editor de texto: este arquivo é parte de um documento maior)
% para saber mais: https://tex.stackexchange.com/q/78101

\chapter{Resultados}
\label{chap:resultados}

São apresentadas nesta seção as previsões realizadas pelo modelo que melhor 
performou neste trabalho: a rede neural recorrente especificada no tópico 
anterior.

No gráfico abaixo, se encontra a distribuição dos deltas percentuais 
que indicam uma tendência do modelo de superestimar a demanda por cimento.



Foram selecionados os estados de São Paulo (SP), Minas Gerais (MG), 
Ceará (CE) e Paraná (PR) para ilustrar o comportamento do modelo. 
No gráfico abaixo, cada linha contínua representa a previsão realizada 
pelo modelo para um estado em específico, a linha pontilhada, por sua vez, 
corresponde à real demanda por cimento no estado correspondente.


Pode-se notar que a previsão realizada pelo modelo não coincide perfeitamente 
com os dados de consumo, comportamento esperado uma vez que é comum que haja 
erro de medição, ruídos ou comportamentos anormais nos dados. 