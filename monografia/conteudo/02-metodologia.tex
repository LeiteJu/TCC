%!TeX root=../tese.tex
%("dica" para o editor de texto: este arquivo é parte de um documento maior)
% para saber mais: https://tex.stackexchange.com/q/78101

\chapter{Metodologia}

\section{Dados}

O objetivo deste trabalho é prever o consumo mensal de cimento nos estados 
do Brasil. Esse indicador foi obtido do Sindicato Nacional da Indústria do 
Cimento e apresentava granularidade mensal para cada um dos estados do 
país.

\begin{table}[H]
    \centering
    \caption{Indicadores de crescimento econômico}
    \begin{tabular}{llll}
        \toprule
        Indicador                   & Fonte & Período disponível & Granularidade         \\
        \midrule
        PIB a preços constantes     & IBGE  & 1983 até 2019      & anual por estado      \\
        PIB a preços de mercado     & IBGE  & 1985 ate 2019      & anual por estado      \\
        PIB per capita              & IBGE  & 1985 até 2019      & anual por estado      \\
        PIB da construção civil     & IBGE  & 1985 até 2019      & anual por estado      \\
        Desemprego                  & IBGE  & 1991 até 2022      & anual por estado até 2014 e mensal para o Brasil a partir de 2012      \\
        \bottomrule
    \end{tabular}
\end{table}

\begin{table}[H]
    \centering
    \caption{Indicadores de inflação e política monetária}
    \begin{tabular}{llll}
        \toprule
        Indicador                   & Fonte & Período disponível & Granularidade         \\
        \midrule
        IPCA                        & IBGE  & 1981 até 2021      & mensal para o Brasil      \\
        INCC                        & FGV   & 1980 ate 2021      & mensal para o Brasil      \\
        IGP                         & FGV   & 1944 até 2021      & mensal para o Brasil      \\
        SELIC                       & IBGE  & 1986 até 2022      & mensal para o Brasil      \\
        \bottomrule
    \end{tabular}
\end{table}

\begin{table}[H]
    \centering
    \caption{Indicadores de inflação e política fiscal}
    \begin{tabular}{llll}
        \toprule
        Indicador                   & Fonte & Período disponível & Granularidade         \\
        \midrule
        NFSP                        & BACEN  & 1991 até 2022      & mensal para o Brasil      \\
        estoque de dívida pública   & IPEA   & 1947 ate 2019      & anual para o Brasil      \\
        \bottomrule
    \end{tabular}
\end{table}

\begin{table}[H]
    \centering
    \caption{Indicadores sociais}
    \begin{tabular}{llll}
        \toprule
        Indicador                   & Fonte & Período disponível & Granularidade         \\
        \midrule
        População                   & IBGE   & 1991 até 2021      & anual por estado      \\
        IDH                         & IBGE   & 1991 ate 2017      & irregular      \\
        \bottomrule
    \end{tabular}
\end{table}


\begin{table}[H]
    \centering
    \caption{Indicadores de construção civil}
    \begin{tabular}{llll}
        \toprule
        Indicador                   & Fonte & Período disponível & Granularidade         \\
        \midrule
        Produção mensal de cimento  & SNIC  & 2003 até 2022      & mensal por estado      \\
        Valor médio do cimento Portland em reais por quilograma   & IPEA   & 1947 ate 2019      & anual para o Brasil      \\
        \bottomrule
    \end{tabular}
\end{table}

Com o objetivo de direcionar a estratégia de preparação de dados
foi realizada uma análise exploratória de cada um dos dados de 
entrada e da variável resposta. A partir dessa análise, 
optou-se por utilizar os dados de 2003 até 2019 para o estudo.

Foram adotadas estratégias para garantir dados na granularidade
mensal e por estado. Caso os indicadores apresentassem granularidade anual, 
o valor foi dividido por 12 de modo a obter a média mensal, já caso a granularidade
fosse a nível de Brasil, o valor apresentado foi repetido para todos os 
estados.


A estratégia utilizada para lidar com dados faltantes foi, sempre que possível,
repetir o valor anterior que estava disponível nos dados de entrada para
preencher a ocorrência. Contudo, alguns indicadores não apresentavam 
valores mais antigos, então foi usado um valor não presente no intervalo
de dados de entrada para marcar como nulo.

Além disso, tomou-se um cuidado para evitar que a previsão fosse 
realizada com os dados do mês anterior ou do ano anterior no caso dos
indicadores anuais. Deslocou-se, portanto, os dados de entrada 
para frente em um ocorrência de modo a associar os dados de um 
mês com o consumo no mês seguinte.\footnote{os dados correspondentes 
a, por exemplo, fevereiro de 2004 estão relacionados ao consumo de 
cimento em março de 2004, com o objetivo de propor um cenário mais pertinente, 
uma vez que o objetivo do projeto é prever a demanda por cimento no mês seguinte 
em um estado a partir dos dados do mês atual e, eventualmente, dos anteriores.
}

Por fim, o estado correspondente à medição foi usado como dado de entrada. 
Como os modelos de inteligência artificial aceitam apenas caracteres numéricos,
utilizou-se o método de codificação \textit{one hot} para criar 27 colunas, uma
para cada estado, nas quais o valor é 1 quando a linha possui dados daquele estado 
e é 0 caso contrário.


    \section{Avaliação de performance}

    Para comparar a eficiência dos modelos mede-se os erros de 
    cada previsão, ou seja, a distância entre o valor previsto 
    pelo algoritmo e o valor do dado real. Neste trabalho, 
    utilizou-se as seguintes métricas estatísticas para 
    mensurar o desempenho: \textit{mean absolute error} (MAE),
    \textit{root mean square  error} (RMSE) e \textit{mean 
    absolute percentage error} (MAPE). Além disso, foi utilizado
    o delta percentual ($\Delta$) para avaliar se o modelo tende 
    a subestimar ou superestimar o valor previsto, se é otimista
    ou pessimista.

\subsection{Mean absolute error (MAE)}

    O MAE, sigla do inglês para \textit{mean absolute error}
    ou média do erro absoluto mede o erro absoluto de cada previsão
    e é dado por:\cite{forecast-evaluation-ds}

    \begin{equation}
        MAE = \frac{\sum_{i=1}^n |\hat{y}_i - y_i|}{n}
    \end{equation}

\subsection{Root mean squared error (RMSE)}

    A RMSE, sigla para \textit{root mean squared  error} é
    semelhante à MAE, contudo eleva os erros ao quadrado antes de 
    somá-los e tira aa raiz logo depois. A RMSE é, por tanto, 
    mais sensível a \textit{outliers}.\cite{forecast-evaluation-ds}

    \begin{equation}
        RMSE = \sqrt{\frac{\sum_{i=1}^n (\hat{y}_i - y_i)^2}{n}}
    \end{equation}

\subsection{Mean absolute percentage error (MAPE)}

    Foi utilizada também a MAPE, \textit{Mean absolute
    percentage error}, para mensurar a escala do erro em 
    relação ao tamanho das medições.

    \begin{equation}
        MAPE=\sum_{t=1}^n\left|\frac{y_t-\hat{y}_t}{y_t}\right|
    \end{equation}

\subsection{Delta percentual}

O delta percentual, $\Delta$, é utilizado para mensurar se o 
modelo apresenta tendência de subestimar ou superestimar a variável, se 
é otimista ou pessimista.

\begin{equation}
    \Delta = \frac{\hat{y_i} - y_i}{y_i}
\end{equation}