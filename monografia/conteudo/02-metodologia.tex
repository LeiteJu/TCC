%!TeX root=../tese.tex
%("dica" para o editor de texto: este arquivo é parte de um documento maior)
% para saber mais: https://tex.stackexchange.com/q/78101

\chapter{Metodologia}

Os modelos recebem como entrada dados econômicos obtidos do Instituto Brasileiro de Geografia e Estatística (IBGE), Fundação Getúlio Vargas (FGV), Instituto de Pesquisa Econômica Aplicada, entre outros. Sobre crescimento econômico, foram utilizados: PIB do estado, PIB per capita, população , PIB da construção civil e desemprego. Já para política fiscal, utilizou-se: Necessidade de Financiamento do Setor Público (NFSP) e Estoque da Dívida Pública. Para mensurar a inflação foram utilizados Índice de Preços ao Consumidor Aplicado (IPCA), Índice Nacional de Custo da Construção (INCC) e Índice Geral de Preço (IGP). O Índice de Desenvolvimento Humano (IDH) foi utilizado como indicador social. Finalmente, a produção de cimento e o preço do saco de 50kg, da tonelada e do quilograma de cimento foram utilizados como indicadores da construção civil.

% \begin{table}[H]
%     \begin{tabular} {\colwidth } { |c|c| }
%         \hline
%         Categoria   & Indicadores  \\
%         \hline
%         Crescimento econômico  & \makecell{PIB do estado, PIB per capita, população , PIB da construção \\ civil e desemprego} \\
%         \hline
%         Política Fiscal  & \makecell{Necessidade de Financiamento do Setor Público (NFSP) \\ e Estoque da Dívida Pública}  \\
%         \hline
%         Inflação & \makecell{Índice de Preços ao Consumidor Aplicado (IPCA), Índice \\ Nacional de Custo da Construção (INCC)  e Índice \\ Geral  de Preço (IGP)} \\
%         \hline
%         Política monetária   & Taxa SELIC  \\
%         \hline
%         Indicadores sociais  & Índice de Desenvolvimento Humano (IDH)  \\
%         \hline
%         \makecell{Indicadores da \\ Construção Civil} & \makecell{Produção de cimento, preço do saco de 50kg, da tonelada \\ e do quilograma  de cimento} \\
%         \hline
%     \end{tabular}
% \end{table}

% % pensar em não ser tão bruta na análise exploratória... A fim de entender os dados de entrada e direcionar a limpeza. ...
% % qual era o periodo que tinha disponível e pq optou por fazer de 2003 a 2019
    A partir da análise exploratória dos dados de entrada, optou-se por utilizar dados de 2003 até 2019, com granularidade mensal, para realizar o estudo. Se os indicadores apresentavam granularidade anual, calculou-se a divisão da medição de cada ano por 12 meses para obter a média mensal. Além disso, caso o dado estivesse disponível apenas a nível de Brasil, ao invés de por estado, o valor da medição da União foi utilizado para todos os estados.
    
    
% pensar em adicionar mais detalhes -> quando fez um ou outro (em alguns casos)
    A estratégia utilizada para lidar com dados faltantes nas variávies de entrada foi repetir o valor da ocorrência anterior ou marcar a entrada com um valor não presente no intervalo de dados, a exemplo de marcar "-1" como valor da produção mensal de cimento em um determinado mês em um estado específico.


    Finalmente, os dados utilizados como entrada foram deslocados um mês à frente ou um ano, no caso dos indicadores anuais, em relação aos dados de consumo. Dessa forma, os dados correspondentes a, por exemplo, fevereiro de 2004 estão relacionados ao consumo de cimento em março de 2004, com o objetivo de propor um cenário mais pertinente, uma vez que o objetivo do projeto é prever a demanda por cimento no mês seguinte em um estado a partir dos dados do mês atual e, eventualmente, dos anteriores.
