%!TeX root=../tese.tex
%("dica" para o editor de texto: este arquivo é parte de um documento maior)
% para saber mais: https://tex.stackexchange.com/q/78101

\chapter{Metodologia}

% falar sobre validação cruzada
% falar que foi em python, libs, 
% dados amostrados ao longo do tempo => não misturamos 
% validação cruzadas 
% n modelo => falar melhores

% melhorar.. só usar isso?

Nesta seção é feito um detalhamento dos dados utilizados, que 
abrange fonte, preparação e pré-processamento. Também são descritas as 
tecnologias utilizadas na implementação, as métricas estatísticas 
usadas para mensurar o desempenho dos modelos e o método de validação
cruzada aplicado para avaliar os modelos.

\section{Tecnologias utilizadas}

Este projeto foi implementado na linguagem Python e 
utilizando o ambiente de desenvolvimento interativo fornecido pelo Jupyter.
Para realizar a preparação dos dados foram utilizadas as bibliotecas Pandas e 
Numpy e para o treinamento e avaliação dos modelos utilizou-se a Tensorflow
e a Scikit-learn. A maior parte dos gráficos foram gerados por meio das bibliotecas 
Seaborn e Matplotlib, contudo alguns foram contruídos com a plataforma
Microsoft Excel.

Os códigos utilizados no projeto estão disponíveis em \url{https://github.com/LeiteJu/TCC}.

\section{Dados}
\label{sec:dados}

Os algoritmos de aprendizado de máquina utilizam dados para fazer 
previsões. 
Neste trabalho, optou-se por utilizar dados de 2003 até 2019 para treinamento e 
avaliação dos modelos, em virtude da disponibilidade dos dados de consumo mensal 
de cimento por estado. Além disso,
adotou-se a estratégia de utilizar granularidade de dados
mensal e por estado com o intuito de aumentar a quantidade de entradas disponíveis 
para treinamento e avaliação dos modelos\footnote{Por exemplo, se fossem usados dados 
anuais a nível de estados da União, haveria 459 \textit{inputs} para os modelos, ao 
utilizar dados mensais por estado, o número de entradas disponíveis aumenta para 5508.}.

Essa estratégia de aumentação de dados visa diminuir o risco de ocorrer 
\textit{underfitting} nos experimentos. O \textit{underfitting} acontece quando o 
modelo não é capaz de aprender com os dados e resulta em altos erros nas etapas de
treinamento e teste (\cite{Goodfellow-et-al-2016}).


\subsection{Fontes}

O modelo utiliza dados econômicos, sociais e da construção
civil para estimar a demanda por cimento. Na tabela
\ref{tab:indicadores}, são
apresentados a fonte, a granularidade e o período em que os dados 
estavam disponíveis.


\begin{table}[H]
    \centering
    \caption{Indicadores utilizados no trabalho}
    \begin{tabular}{llll}
        \toprule
        Dado                   & Fonte & Período disponível & Granularidade         \\
        \midrule
        PIB a preços constantes     
                                    & IBGE\footnote{\label{portal ipea} Dado retirado do portal do Ipeadata em \url{http://www.ipeadata.gov.br/Default.aspx}}  & 1983 até 2019      & anual por estado      \\
        PIB a preços de mercado      & IBGE\footref{portal ipea}  & 1985 ate 2019      & anual por estado      \\
        PIB per capita              & IBGE\footref{portal ipea}  & 1985 até 2019      & anual por estado      \\
        PIB da construção civil      & IBGE\footref{portal ipea}  & 1985 até 2019      & anual por estado      \\
        Desemprego                   & IBGE\footref{portal ipea}  & 1991 até 2022      & irregular \footnote{Havia dados de 1992 até 2014
        com granularidade anual e por estado. A partir de 2012 foram disponibilizados dados mensais a nível de Brasil por conta da 
        Pesquisa Nacional por Amostra de Domicílios (PNAD) Contínua mensal realizada pelo IBGE. Neste trabalho, utilizou-se os dados anuais até 2012
        e, após 2012, os dados provenientes da PNAD Contínua.} \\
        IPCA                        & IBGE\footnote{Dado retirado do IBGE em \url{https://sidra.ibge.gov.br/tabela/1737}}  & 1981 até 2021      & mensal para o Brasil      \\
        INCC                        & FGV\footnote{Dado obtido a partir do portal da FGV em \url{https://www.debit.com.br/tabelas/tabela-completa-pdf.php?indice=incc}}   & 1980 ate 2021      & mensal para o Brasil      \\
        IGP                         & FGV\footref{portal ipea}   & 1944 até 2021      & mensal para o Brasil      \\
        Taxa Selic                  & IBGE\footnote{Dado obtido em \url{https://www.debit.com.br/tabelas/tabela-completa.php?indice=selic}}  & 1986 até 2022      & mensal para o Brasil      \\
        NFSP                        & BACEN\footref{portal ipea}  & 1991 até 2022      & mensal para o Brasil      \\
        Estoque líquido de capital fixo   & IPEA\footref{portal ipea}   & 1947 ate 2019      & anual para o Brasil      \\
        População                   & IBGE\footnote{Dado obtido do portal Base dos Dados em \url{https://basedosdados.org/dataset/br-ibge-populacao}}   & 1991 até 2021      & anual por estado      \\
        IDH                         & IBGE\footref{portal ipea}   & 1991 ate 2017      & irregular\footnote{Os indicadores de IDH (Renda, Longevidade e Educação) estão disponíveis em anos de censo do IBGE (1990, 2000, 2010). Há 
        dados, também, de 2014 a 2017 por conta da PNAD Contínua.}      \\
        Produção mensal de cimento  & SNIC\footnote{\label{cbic} Dados retirados do portal \url{http://www.cbicdados.com.br/menu/materiais-de-construcao/cimento}}  & 2003 até 2022      & mensal por estado      \\
        Valor médio do cimento\footnote{Evolução do valor médio/mediano do cimento Portland 32 em US\$/Tonelada}      & SNIC\footref{cbic}   & 1947 ate 2019      & anual para o Brasil      \\
        Consumo de cimento em ton.  & SNIC\footref{cbic} & 2003 até 2019 & mensal por estado \\
        \bottomrule
    \end{tabular}
    \label{tab:indicadores}
\end{table}
% quem sabe por a legenda no apendice

Na tabela, são utilizadas siglas 
para melhorar a legibilidade.
Em particular IBGE para Instituto Brasileiro
de Geografia e Estatística, FGV para Fundação Getúlio Vargas,
BACEN para Banco Central do Brasil, IPEA para Instituto de 
Pesquisa Econômica Aplicada e SNIC para Sindicato Nacional 
da Indústria do Cimento.
Além disso, utilizou-se abreviação para Produto Interno Bruto (PIB),
Índice Nacional de Preços ao Consumidor Amplo (IPCA),
Índice Nacional de Custo da Construção (INCC), Índice Geral de 
Preços (IGP), Necessidade de Financiamento do Setor Público (NFSP) e 
Índice de Desenvolvimento Humano (IDH).

\subsection{Preparação dos dados}

Com o objetivo de direcionar a estratégia de preparação de dados
foi realizada uma análise exploratória dos dados de 
entrada e da variável resposta. 
Destacou-se nessa análise, a alta
taxa de variação dos dados de entrada. Alguns indicadores, como o PIB
da construção civil apresentam
o desvio padrão maior que o valor médio do indicador. Dessa 
forma, está presente nos dados um grande número de \textit{outliers},
em especial ao analisar o Brasil como um todo, por conta da 
forte diferença entre as regiões do Brasil. 


\textit{Outliers} são observações discrepantes do restante dos Dados
que podem interferir no processo de previsão (\cite{outliers} e \cite{tukey77}).
Para ilustrar melhor o alto volume de \textit{outliers} presentes
nos dados, utilizou-se gráficos \textit{boxplot}, como mostrado na 
figura abaixo.

\begin{figure}[H]
    \centering
    \includegraphics[width=8cm]{../figuras/explicacao-boxplot.png}
    \caption{Gráfico \textit{boxplot} (\cite{explicacao-boxplot})}
    \label{fig:boxplot}
\end{figure}

O \textit{boxplot} \cite{boxplot} é
uma técnica estatística utilizada na análise exploratória
os dados para identificar visualmente padrões, como a distribuição
dos dados. Na figura, acima linha em verde corresponde à mediana \footnote{
valor que fica no meio quando os dados estão ordenados ou a média
dos dois valores centrais se o número de pontos de dados for par
\cite{boxplot-stat}} 
dos dados, os limite inferior e o superior ao retângulo representam
o primeiro e terceiro quartil\footnote{O primeiro e terceiro 
quartis também representam a mediana dos valores superiores
à mediana dos dados e a mediana dos valores inferiores à mediana. Então metade
dos dados está contida dos quadrados nas imagens}, respectivamente. 

\begin{figure}[H]
    \centering
    \label{fig:boxplot_all}
    \begin{subfigure}{5 cm}
        \centering
        \includegraphics[width=5cm]{../figuras/graficos/boxplot-pib-cc.png}
        \caption{Gráfico \textit{boxplot} do Brasil}
    \end{subfigure}
    \hfill
    \begin{subfigure}{5cm}
        \centering
        \includegraphics[width=5cm]{../figuras/graficos/boxplot-pib-cc-se.png}
        \caption{Gráfico \textit{boxplot} do Sudeste}
    \end{subfigure}
    \hfill
    \begin{subfigure}{5cm}
        \centering
        \includegraphics[width=4.8cm]{../figuras/graficos/boxplot-pib-cc-n.png}
        \caption{Gráfico \textit{boxplot} do Norte}
    \end{subfigure}
\end{figure}

Conforme explicado na figura \ref{fig:boxplot}, a linha em 
laranja nos gráficos acima representa a mediana dos dados,
o retângulo que envolve a mediana é delimitado pelo primeiro e 
terceiro quartil e retém metade central das amostras. Os valores
reoresentados por pontos vermelhos acima ou abaixo dos limites
superiores ou inferiores são \textit{outliers}. Observa-se,
então, que pode-se obter uma significativa redução no número de 
amostras com \textit{outliers} ao separar a análise por região.\cite{boxplot}

Analisou-se, também, a correlação entre as variáveis de entrada e observou-se
alta correlação entre os indicadores de PIB do estado, PIB da construção
civil e população. Além disso, há alta correlação entre os três indicadores
de IDH, além do preço do saco de cimento e o preço do kilograma, como pode-se
validar na figura \ref{fig:matriz-corr}.

\begin{figure}[H]
    \centering
    \includegraphics[width=13cm]{../figuras/graficos/matriz-corr.png}
    \caption{Matriz de correlação}
    \label{fig:matriz-corr}
\end{figure}

Foram adotadas estratégias para garantir dados na granularidade
mensal e por estado. Caso os indicadores apresentassem granularidade anual, 
o valor foi dividido por 12 de modo a obter a média mensal, já caso a granularidade
fosse a nível de Brasil, o valor apresentado foi repetido para todos os 
estados.


A estratégia utilizada para lidar com dados faltantes foi, sempre que possível,
repetir o valor anterior que estava disponível nos dados de entrada para
preencher a ocorrência. Contudo, alguns indicadores não apresentavam 
valores mais antigos, então foi usado um valor não presente no intervalo
de dados de entrada (-1) para marcar como nulo. Os indicadores 
com dados faltantes são, em ordem descrescente de acordo com 
o percentual de dados faltantes: produção de cimento, valor 
médio do quilo, saco e da tonelada e cimento e desemprego.

Além disso, tomou-se um cuidado para evitar que a previsão fosse 
realizada com os dados do mês anterior ou do ano anterior no caso dos
indicadores anuais. Deslocou-se, portanto, os dados de entrada 
para frente em um ocorrência de modo a associar os dados de um 
mês com o consumo no mês seguinte.\footnote{os dados correspondentes 
a, por exemplo, fevereiro de 2004 estão relacionados ao consumo de 
cimento em março de 2004, com o objetivo de propor um cenário mais pertinente, 
uma vez que o objetivo do projeto é prever a demanda por cimento no mês seguinte 
em um estado a partir dos dados do mês atual e, eventualmente, dos anteriores.
}

Por fim, o estado correspondente à medição foi usado como dado de entrada. 
Como os modelos de inteligência artificial aceitam apenas caracteres numéricos,
utilizou-se o método de codificação \textit{one hot} para criar 27 colunas, uma
para cada estado, nas quais o valor é 1 quando a linha possui dados daquele estado 
e é 0 caso contrário.


\subsection{Pré-processamento de dados}
\label{sec:norm_dados}


\subsubsection{Normalização}



O objetivo da normalização dos dados, é garantir que as variáveis de entrada 
estejam na mesma escala. Esse processo é necessário não apenas para realizar melhor comparação entre variáveis 
com diferentes unidades, mas também para o melhor funcionamentos dos algoritmos 
de aprendizado, uma vez que se as \textit{features} estiverem em escalas diferentes
alguns pesos podem ser atualizados mais rápidos que outros. (\cite{Raschka})


A normalização garante que as variáveis de entrada estejam em uma 
escala com as propriedades de uma distribuição normal, ou seja, média ($\mu$)
igual a zero e variância ($\sigma$) igual a um. Dessa forma, a operação 
realizada para aplicar o processo em um entrada $x$ é

\begin{equation}
  z = \frac{x - \mu}{\sigma}
\end{equation}

\subsubsection{\textit{Min-Max scaler}}

O \textit{min-max scaler} transforma os dados de modo que a 
escala dos dados tenha um itnervalo definido, em geral, entre
0 e 1.

\begin{equation}
  X_{norm} = \frac{X - X_{min}}{X_{max} - X_{min}}
\end{equation}




\section{Avaliação de performance}

Para comparar a eficiência dos modelos mede-se os erros de 
cada previsão, ou seja, a distância entre o valor previsto 
pelo algoritmo e o valor do dado real. Neste trabalho, 
utilizou-se as seguintes métricas estatísticas para 
mensurar o desempenho: \textit{mean absolute error} (MAE),
\textit{root mean square  error} (RMSE) e \textit{mean 
absolute percentage error} (MAPE). Além disso, foi utilizado
o delta percentual ($\Delta$) para avaliar se o modelo tende 
a subestimar ou superestimar o valor previsto, se é otimista
ou pessimista.

\subsection{Mean absolute error (MAE)}

    O MAE, sigla do inglês para \textit{mean absolute error}
    ou média do erro absoluto mede o erro absoluto de cada previsão
    e é dado por:\cite{forecast-evaluation-ds}

    \begin{equation}
        MAE = \frac{\sum_{i=1}^n |\hat{y}_i - y_i|}{n}
    \end{equation}

\subsection{Root mean squared error (RMSE)}

    A RMSE, sigla para \textit{root mean squared  error} é
    semelhante à MAE, contudo eleva os erros ao quadrado antes de 
    somá-los e tira a raiz logo depois. A RMSE é, portanto, 
    mais sensível a \textit{outliers}.\cite{forecast-evaluation-ds}

    \begin{equation}
        RMSE = \sqrt{\frac{\sum_{i=1}^n (\hat{y}_i - y_i)^2}{n}}
    \end{equation}

\subsection{Mean absolute percentage error (MAPE)}

    Foi utilizada também a MAPE, \textit{Mean absolute
    percentage error}, para mensurar a escala do erro em 
    relação ao tamanho das medições.

    \begin{equation}
        MAPE=\sum_{t=1}^n\left|\frac{y_t-\hat{y}_t}{y_t}\right|
    \end{equation}

\subsection{Variação percentual}

A variação percentual, $\Delta$, é utilizado para mensurar se o 
modelo apresenta tendência de subestimar ou superestimar a variável, se 
é otimista ou pessimista.

\begin{equation}
    \Delta = \frac{\hat{y_i} - y_i}{y_i}
\end{equation}

