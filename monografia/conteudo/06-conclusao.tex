%!TeX root=../tese.tex
%("dica" para o editor de texto: este arquivo é parte de um documento maior)
% para saber mais: https://tex.stackexchange.com/q/78101

\chapter{Conclusão}
\label{chap:conclusao}

Este trabalho tinha o objetivo de propor uma solução para a demanda da indústria cimenteira 
por um modo bem fundamentado para prever a demanda por cimento nos estados Brasil.
Assim, foi investigado o desempenho de diferentes modelos de aprendizado na 
tarefa de utilizar dados econômicos, sociais, monetários e da construção civil 
para prever o consumo mensal de cimento, os modelos testados, em ordem 
crescente de complexidade, são regressão linear, redes neurais \textit{feed 
forward}, redes recorrentes e redes bidirecionais. 

Ao longo desta monografia foram apresentados os conceitos de aprendizado 
de máquina, regressão linear, redes neurais \textit{feed forward}, redes 
recorrentes e redes bidirecionais. Além disso, descreveu-se a metodologia aplicada, 
o tratamento realizado nos dados, os métodos para mensurar o desempenho dos 
modelos e os experimentos realizados. Foram detalhados, também, as previsões
realizadas pelos modelos e a avaliação dessas.

Evidenciou-se ao longo do trabalho o aumento na assertividade das previsões 
à medida que se aumentava a robustez dos modelos utilizados, em geral a melhora
também foi perceptível nas métricas de erro absoluto. Assim, as redes bidirecionais
são o modelo com melhor desempenho dentre os testados no estudo, com previsões
assertivas que capturaram com precisão a tendência do consumo. Apesar do erro 
associado às previsões, esperado uma vez que o fenômeno que se deseja prever 
é influenciado por fatores além da capacidade do modelo de capturar, foi 
alcançada uma relevante assertividade nas previsões, além de um baixo erro 
absoluto.

Dessa forma, infere-se que o uso de modelos de \textit{machine learning} na indústria 
de cimento para prever a demanda pelo material é promissor e capaz de produzir 
relevantes resultados. Configura-se, portanto, como uma possível solução para 
a deficiência dessa indústria de um modo embasado para estimar o consumo futuro de 
cimento. 

Espera-se que os experimentos, resultados e conclusões realizados 
neste trabalho de conclusão de curso auxiliem futuros pesquisadores a retomar a 
pesquisa deste projeto.
