%!TeX root=../tese.tex
%("dica" para o editor de texto: este arquivo é parte de um documento maior)
% para saber mais: https://tex.stackexchange.com/q/78101

\chapter{Experimentos}

Para cada um dos modelos selecionados foram realizados experimentos
alterando parâmetros e os dados visando melhoria de performance. Os experimentos
realizados variam de acordo como o modelo, como pode ser visto a seguir:

\section{Regressao linear}

A regressão linear é um modelo mais simples de aprendizado de máquina que assume
uma relação linear entre as variáveis de entrada e o alvo da previsão. Foram testados
métodos de normalização e transformação dos dados, além da remoção de variáveis de 
alta correlação e o uso apenas de dados mais recentes (a partir de 2009) para 
realizar a previsão. As veriáveis de alta correlação \ref{sec:dados} que foram removidas são: 
PIB \textit{per capita}, INCC, IGP, taxa Selic, IDH Educação e IDH Longevidade, 
NFSP, preço do saco de cimento e preço da tonelada de cimento.

\begin{table}
    \centering
    \begin{tabular}{llll}
        \toprule
        Experimento & Transformação nos dados     & Período de dados & Remoção de variáveis  \\
        \midrule
        A           & nenhum & de 2003 até 2019            & não~                                     \\
        B           & Standard scaler~            & de 2003 até 2019            & não~ ~                                   \\
        C           & MinMax scaler~ ~            & de 2003 até 2019            & não~ ~ ~                                 \\
        D           & Power Transformer           & de 2003 até 2019            & não                                      \\
        E           & nenhum & de 2003 até 2019            & sim                                      \\
        F           & nenhum & de 2009 até 2019            & não                                      \\
        \bottomrule
    \end{tabular}
    \label{tab:exp-reg-lin}
    \caption{Experimentos realizados na regressão linear}
\end{table}

Na sessão \ref{chap:resultados}, estão apresentados os 
resultados e previsões do modelo de regressão linear com
melhor desempenho dentre os testados no modelo.


\section{Redes neurais MLP}
\section{Redes recorrentes}
\section{regressao linear}