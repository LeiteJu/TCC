%!TeX root=../tese.tex
%("dica" para o editor de texto: este arquivo é parte de um documento maior)
% para saber mais: https://tex.stackexchange.com/q/78101

\chapter{Experimentos}

Para cada um dos modelos selecionados foram realizados experimentos
alterando parâmetros e os dados com o objetivo de obter um melhor 
desempenho. O treino dos modelos foram realizados com dados 
de 2003 até junho de 2017,  os testes, por sua vez, foram realizados com dados de 
julho de 2017 até dezembro de 2019, dessa forma 85\% do conjunto 
de dados foi usado para treino e 15\% para teste. Além disso, 
cuidou-se para que não houvesse embaralhamento de dados no 
processo de separação entre treino e teste, uma vez trata-se 
de dados temporais. Para cada modelo foram 
alterados diferentes parâmetros.

\section{Regressao linear}

A regressão linear é um modelo simples de aprendizado de máquina que visa 
construir uma função linear entre as variáveis de entrada e o alvo da previsão. 
Por tratar-se de um modelo mais simples, é utilizado 
neste estudo como base para comparar o desempenho dos
outros modelos. Foram testados métodos de normalização  e transformação de 
dados, além da 
remoção de variáveis de alta correlação conforme 
descrito na tabela \ref{tab:exp-reg-lin}. 

\begin{table}
    \centering
    \begin{tabular}{llll}
        \toprule
        Experimento & Transformação nos dados     & Período de dados & Remoção de variáveis  \\
        \midrule
        A           & nenhum & de 2003 até 2019            & não~                                     \\
        B           & \textit{Standard scaler}~            & de 2003 até 2019            & não~ ~                                   \\
        C           & \textit{MinMax scaler}~ ~            & de 2003 até 2019            & não~ ~ ~                                 \\
        D           & \textit{Power Transformer}           & de 2003 até 2019            & não                                      \\
        E           & nenhum & de 2003 até 2019            & sim\footnote{As variáveis removidas foram: PIB \textit{per capita}, INCC, IGP, taxa Selic, IDH Educação e IDH Longevidade, 
        NFSP, preço do saco de cimento e preço da tonelada de cimento}                                      \\
        \bottomrule
    \end{tabular}
    \label{tab:exp-reg-lin}
    \caption{Experimentos realizados na regressão linear}
\end{table}

O desempenho do modelo em cada um dos experimentos 
encontra-se na 
tabela \ref{tab:res_reg_lin}:

\begin{table}[H]
    \begin{tabular}{llll}
        \toprule
        Experimento & MAE     & RMSE    & MAPE \\
        \midrule
        A           & 35148.6 & 55391.1 & 0.6  \\
        B           & 35148.6 & 55391.1 & 0.6  \\
        C           & 35148.6 & 55391.1 & 0.6  \\
        D           & 49073.6 & 67446.7 & 1.05 \\
        E           & \textbf{26435.6} & \textbf{38970.6} & \textbf{0.32} \\
        \bottomrule
    \end{tabular}
    \caption{Desempenho dos modelos de regressão linear}
    \label{tab:res_reg_lin}
\end{table}

Observa-se que não houve alteração na performance ao normalizar 
os dados com \textit{standard scaler} ou \textit{minmax scaler},
ao utilizar \textit{power transformer}, contudo, houve piora 
no desempenho. Ressalta-se a grande melhoria nas métricas ao 
remover variáveis de alta correlação tanto nas métricas de erro 
quantitativo, como o MAE e RMSE, quanto proporcionalmente ao 
atingir 32\% de erro percentual médio.

Foi realizado o teste de remover variáveis de alta correlação, 
pois segundo \cite{corr_reg_lin}, esse tipo de \textit{feature}
pode ter efeitos ruins no modelo como: aumentar a variabilidade
dos coeficientes em relação à amostra, ou seja, os coeficientes 
calculados pelo modelo podem ter um alto grau de variação de uma 
amostra para outra.

Na sessão \ref{chap:resultados}, estão apresentados os 
resultados e previsões do modelo de regressão linear com
melhor desempenho dentre os testados no modelo, ou seja, o
experimento E, sem 
normalização de dados e removendo \textit{features} com alta 
correlação.

\section{Redes neurais \textit{Feed Forward}}

As redes neurais \textit{feed forward} podem ser construídas
com várias arquiteturas e configurações. Neste estudo, testou-se
alterar a quantidade de camadas da rede, o número neurônios em 
cada camada, a função de ativação utilizada, além da quantidade
de \textit{epoch}\footnote{Uma \textit{epoch} é uma passada 
pelos dados de entrada \cite{dl-oreilly}} utilizada no treinamento.



\section{Redes recorrentes}
\section{regressao linear}