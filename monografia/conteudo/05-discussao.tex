%!TeX root=../tese.tex
%("dica" para o editor de texto: este arquivo é parte de um documento maior)
% para saber mais: https://tex.stackexchange.com/q/78101

\chapter{Discussão}
\label{chap:discussao}

Nesta dissertação, a diferença entre 1 e o valor da MAPE de um 
modelo foi empregada como métrica de assertividade, com o propósito de 
mensurar o grau de acerto das previsões. 

Durante a análise dos experimentos realizados, foi possível observar que as previsões se tornaram mais assertivas à medida que se elevou a complexidade do modelo. Essa melhora nas previsões indicou que não houve sintomas de falta de dados e era esperada, visto que as redes \textit{feed forward} acrescentaram à regressão linear
a capacidade de reconhecer padrões não lineares nos dados por meio do 
processamento matemático realizado nas camadas de neurônios. As redes 
recorrentes, de maneira análoga, adicionaram às redes \textit{feed forward} a capacidade de identificar contexto histórico e sequências e, por fim, as redes bidirecionais agregaram às recorrentes a habilidade de avaliar as entradas nos dois sentidos das sequências. 

A regressão linear é o modelo mais simples empregado neste estudo e, apesar
de apresentar as previsões menos exatas entre os modelos, obteve 68\% de 
assertividade. Seu erro absoluto médio foi de 26.4 mil toneladas, uma taxa de erro aceitável ao levar em consideração que a média de consumo mensal de cimento no Brasil é de 151.2 mil toneladas.

O processamento mais robusto das redes \textit{feed forward}, conferido pelas 
camadas de neurônios, resultou em uma assertividade de 73\% em média, um aumento 
de 5\% em relação à regressão linear. Essa melhora, contudo, não se refletiu 
no erro absoluto que foi de 26.9 mil toneladas e houve um provável
aumento nos \textit{outliers} $-$ amostras cuja previsão foi distante do valor 
real $-$ indicado pelo aumento da RMSE.

A capacidade de analisar contexto histórico garantida às redes recorrentes pela 
arquitetura complexa teve como consequência uma assertividade de 81\%, uma melhora 
de 8\% em comparação com as redes \textit{feed forward} e de 13\% em comparação 
com a regressão linear. Do mesmo modo, o impacto se refletiu na redução do  erro absoluto para 22.7 mil toneladas, ou seja, a rede apresentou um melhor desempenho 
tanto ao considerar o valor absoluto do erro, quanto 
ao comparar a proporção do erro em relação ao resultado que o modelo pretendia prever.

Por fim, as duas redes combinadas conferiram às redes bidirecionais uma 
assertividade de 83\% em média, um aumento de 2\% em relação às redes recorrentes e de
15\% em relação à regressão linear. Além disso, o erro absoluto foi o menor entre 
os modelos testados no estudo $-$ 19.1 mil toneladas $-$, e houve menor 
incidência de previsões muito discrepantes do valor real, uma vez houve queda
na RMSE. Esse modelo também apresentou uma boa capacidade de generalização, 
haja vista que apresentou 12.6\% de erro no conjunto de treino e 17.4\% no de teste,
ou seja, um baixo \textit{overfitting}.

Assim, a capacidade de analisar contexto histórico nos dois sentidos das redes
neurais bidirecionais configurou-se como diferencial e garantiu ao modelo o 
melhor desempenho entre os testados. Ainda que as previsões desse modelo 
não coincidam exatamente com os dados reais de consumo, esse comportamento é 
esperado, uma vez que esse indicador é sujeito a 
influências externas não mensuráveis pelo modelo, como políticas governamentais
em um estado específico, ocorrências climáticas, grandes obras da indústria
privada, entre outros fatores.

