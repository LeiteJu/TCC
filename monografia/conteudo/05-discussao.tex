%!TeX root=../tese.tex
%("dica" para o editor de texto: este arquivo é parte de um documento maior)
% para saber mais: https://tex.stackexchange.com/q/78101

\chapter{Discussão}
\label{chap:discussao}

Neste trabalho, para comparar o desempenho dos modelos, 
 será utilizada a diferença entre 1 e o valor da MAPE de um 
modelo como métrica de assertividade, com o objetivo de mensurar o grau 
de acerto das previsões.

Na análise dos experimentos realizados, é possível observar que à medida que 
se eleva a complexidade do modelo, mais assertivas as previsões se mostram,
como o esperado. As redes \textit{feed forward} acrescentam à regressão linear
a capacidade de reconhecer padrões não lineares nos dados por meio do 
processamento matemático realizado nas camadas de neurônios. As redes neurais 
recorrentes, por sua vez, adicionam a capacidade de identificar contexto 
histórico e sequências. Por fim, as redes bidirecionais agregam a habilidade 
de fazer análises nos dois sentidos das sequências.

A regressão linear é o modelo mais simples utilizado neste trabalho e, apesar
de apresentar as previsões menos exatas entre os modelos, obteve 68\% de 
assertividade. O erro absoluto médio desse modelo é de 26.4 mil toneladas
e ainda que pareça um valor alto, em comparação com as 151.2 mil toneladas de cimento
consumidas em média por mês no Brasil, representa uma taxa de erro aceitável.

O processamento mais robusto das redes \textit{feed forward} em virtude das 
camadas de neurônios resultou em uma assertividade de 73\% em média, um aumento 
de 5\% em relação à regressão linear. Essa melhora, contudo, não se refletiu 
no erro absoluto que foi de 26.9 mil toneladas, houve, também, um provável
aumento nos \textit{outliers} $-$ amostras cuja previsão foi distante do valor 
real $-$ indicado pelo aumento da RMSE.

A capacidade de analisar contexto histórico garantida às redes recorrentes pela 
arquitetura complexa teve como consequência uma assertividade de 81\%, uma melhora 
de 8\% em comparação com as redes \textit{feed forward} e de 13\% em comparação 
com a regressão linear. O impacto também se refletiu no erro absoluto que 
reduziu para 22.7 mil toneladas, ou seja, a rede apresentou um melhor desempenho 
tanto ao considerar o valor absoluto do erro, quanto ao comparar o tamanho do 
erro com o tamanho do resultado que o modelo pretendia prever.

Por fim, as duas redes combinadas conferiram às redes bidirecionais uma 
assertividade de 83\% em média, um aumento de 2\% em relação às redes recorrentes e de
15\% em relação à regressão linear. O erro absoluto também foi o menor entre 
os modelos testados no estudo, 19.1 mil toneladas, além disso houve menor 
incidência de previsões muito discrepantes do valor real, uma vez houve queda
na RMSE. Esse modelo também apresenta uma boa capacidade de generalização, 
já que apresenta 12.6\% de erro no conjunto de treino e 17.4\% no de teste,
ou seja, um baixo \textit{overfitting}.

Assim, a capacidade de analisar contexto histórico nos dois sentidoos das redes
neurais bidirecionais configura-se como diferencial e garante ao modelo o 
melhor desempenho entre os modelos testados. Ainda que as previsões desse modelo 
não coincidam exatamente com os dados reais de consumo, esse comportamento é 
esperado, uma vez que esse indicador é sujeito a 
influências externas não mensuráveis pelo modelo, como políticas governamentais
em um estado específico, ocorrências climáticas, grandes obras da indústria
privada, entre outros fatores.