%!TeX root=../tese.tex
%("dica" para o editor de texto: este arquivo é parte de um documento maior)
% para saber mais: https://tex.stackexchange.com/q/78101

% As palavras-chave são obrigatórias, em português e em inglês, e devem ser
% definidas antes do resumo/abstract. Acrescente quantas forem necessárias.
\palavrachave{Cimento}
\palavrachave{Consumo de Cimento}
\palavrachave{Economia Brasileira}
\palavrachave{\textit{Machine Learning}}
\palavrachave{\textit{Neural Networks}}
\palavrachave{regressão linear}
\palavrachave{inteligência artificial}

\keyword{cement}
\keyword{cement consumption}
\keyword{machine learning}
\keyword{deep learning} 
\keyword{neural networks}
\keyword{linear regression}
\keyword{artificial intelligence}

% O resumo é obrigatório, em português e inglês. Estes comandos também
% geram automaticamente a referência para o próprio documento, conforme
% as normas sugeridas da USP.
\resumo{
O cimento é um dos principais insumos da indústria da construção civil que é uma 
das maiores impulsionadoras da economia brasileira, tendo isso em mente, a indústria 
cimenteira apresenta a demanda por um método bem fundamentado para prever o 
consumo do cimento no Brasil, para melhor direcionar estratégias e investimentos. Este trabalho, 
então, propõe o uso de \textit{machine learning} para prever o 
consumo de cimento no país, além de investigar o desempenho dos modelos para 
identificar o mais eficiente para a tarefa. Como se verá a seguir, foram testadas  regressão linear, 
redes \textit{feed forward}, redes recorrentes e redes bidirecionais, além disso, durante o trabalho foi realizada
uma preparação dos dados para garantir a granularidade necessária, tratar
de valores faltantes e assegurar que foram utilizados apenas dados do passado ao 
realizar a previsão. Por fim, nos experimentos realizados de forma a identificar o modelo com 
melhor desempenho, destaca-se o aumento na assertividade das previsões $-$ mensurada
a partir de métricas de erro absoluto e percentual $-$  à medida que 
se eleva a complexidade do modelo. Assim, o trabalho em questão investiga se é possível prever a demanda por 
cimento com assertividade ao utilizar modelos de \textit{machine learning}.}
 
\abstract{Cement is one of the main inputs of the construction industry, a major driver
of the Brazilian economy. In this context, the cement industry has the demand for a well
grounded way to predict consumption in Brazil, to better direct strategy and investments. This
work, then, proposes the use of machine learning to predict cement consumption in the country, in addition to
investigate the performance of the models to identify the most efficient one for the task. The models receive
as input economic, monetary, social and civil construction indicators corresponding to a state
in a specific month and have the objective of predicting cement consumption in the following month in that state. Were
tested linear regression, feed forward networks, recurrent networks and bidirectional networks. It was carried out during
work a data preparation to ensure the necessary granularity, handle missing values and
ensure that only past data was used when performing the forecast. In the experiments carried out
to identify the model with better performance, the increase in the assertiveness of predictions stands out $-$
measured from absolute and percentage error metrics $-$ as the complexity of the
model increases. The most assertive model among those tested in this work are the bidirectional neural networks, which were able to
to accurately capture the evolution trend of cement consumption in São Paulo, in addition to carrying out the
more assertive predictions among the models, 84.4\% of the model's predictions had a percentage error
less than 30\% and the mean absolute percentage error was 17\%. Thus, the work confirms that it is possible
predict the demand for cement quite assertively by using machine learning models and
that the implementation of this strategy is viable in industries. 
}
