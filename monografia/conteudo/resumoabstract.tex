%!TeX root=../tese.tex
%("dica" para o editor de texto: este arquivo é parte de um documento maior)
% para saber mais: https://tex.stackexchange.com/q/78101

% As palavras-chave são obrigatórias, em português e em inglês, e devem ser
% definidas antes do resumo/abstract. Acrescente quantas forem necessárias.
\palavrachave{cimento}
\palavrachave{consumo de cimento}
\palavrachave{aprendizado de máquina}
\palavrachave{aprendizado profundo}
\palavrachave{redes neurais}
\palavrachave{regressão linear}
\palavrachave{inteligência artificial}

\keyword{cement}
\keyword{cement consumption}
\keyword{machine learning}
\keyword{deep learning}
\keyword{neural networks}
\keyword{linear regression}
\keyword{artificial intelligence}

% O resumo é obrigatório, em português e inglês. Estes comandos também
% geram automaticamente a referência para o próprio documento, conforme
% as normas sugeridas da USP.
\resumo{
O cimento é um dos principais insumos da indústria da construção civil, um  
grande impulsionador da economia brasileira. Nesse contexto, indústria 
cimenteira apresenta a demanda por um método bem fundamentado para prever o 
consumo no Brasil, para melhor direcionar estratégia e investimentos. Este trabalho, 
então, propõe o uso de aprendizado de máquina para prever o 
consumo de cimento no país, além de investigar o desempenho dos modelos para 
identificar o mais eficiente para a tarefa. Os modelos recebem como entrada 
indicadores econômicos, monetários, sociais e da construção civil correspondentes a 
um estado em um mês específicos e têm o objetivo de prever o consumo de cimento 
no mês seguinte nesse estado. Foram testadas  regressão linear, 
redes \textit{feed forward}, redes recorrentes e redes bidirecionais. Foi realizada
durante o trabalho 
uma preparação dos dados para garantir a granularidade necessária, tratar
de valores faltantes e assegurar que foram utilizados apenas dados do passado ao 
realizar a previsão. Nos experimentos realizados para identificar o modelo com 
melhor desempenho, destaca-se o aumento na assertividade das previsões $-$ mensurada
a partir de métricas de erro absoluto e percentual $-$  à medida que 
se eleva a complexidade do modelo. O modelo mais assertivo dentre os testados 
neste trabalho é a rede neural bidirecional, que foi capaz de capturar a tendência 
de evolução do consumo de cimento em São Paulo com precisão, além de realizar 
as previsões mais assertivas dentre os modelos, 84.4\% das previsões do modelo
apresentaram erro percentual inferior a 30\% e a média do erro percentual absoluto 
foi de 17\%. Assim, o trabalho constata que é possível prever a demanda por 
cimento com bastante assertividade ao utilizar modelos de aprendizado de máquina 
e que a implementação dessa estratégia é viável nas indústrias.}

\abstract{
Elemento obrigatório, elaborado com as mesmas características do resumo em
língua portuguesa. De acordo com o Regimento da Pós-Graduação da USP (Artigo
99), deve ser redigido em inglês para fins de divulgação. É uma boa ideia usar
o sítio \url{www.grammarly.com} na preparação de textos em inglês.
Text text text text text text text text text text text text text text text text
text text text text text text text text text text text text text text text text
text text text text text text text text text text text text text text text text
text text text text text text text text text text text text.
Text text text text text text text text text text text text text text text text
text text text text text text text text text text text text text text text text
text text text.
}
