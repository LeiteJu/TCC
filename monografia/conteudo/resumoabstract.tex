%!TeX root=../tese.tex
%("dica" para o editor de texto: este arquivo é parte de um documento maior)
% para saber mais: https://tex.stackexchange.com/q/78101

% As palavras-chave são obrigatórias, em português e em inglês, e devem ser
% definidas antes do resumo/abstract. Acrescente quantas forem necessárias.
\palavrachave{Cimento}
\palavrachave{Consumo de Cimento}
\palavrachave{Economia Brasileira}
\palavrachave{Construção Civil}
\palavrachave{Aprendizado de máquina}
\palavrachave{Redes Neurais}
\palavrachave{Regressão Linear}
\palavrachave{Inteligência Artificial}
\palavrachave{Previsão}

\keyword{Cement}
\keyword{Cement Consumption}
\keyword{Brazilian Economy}
\keyword{Machine Learning}
\keyword{Civil Construction} 
\keyword{Machine Learning}
\keyword{Neural Networks}
\keyword{Linear Regression}
\keyword{Artificial Intelligence}
\keyword{Forecasting}

% O resumo é obrigatório, em português e inglês. Estes comandos também
% geram automaticamente a referência para o próprio documento, conforme
% as normas sugeridas da USP.
\resumo{
O cimento é um dos principais insumos da indústria da construção civil que é uma 
das maiores impulsionadoras da economia brasileira. A indústria 
cimenteira apresenta a demanda por um método bem fundamentado para prever o 
consumo do cimento no Brasil, para melhor direcionar estratégias e investimentos. Este trabalho, 
então, propõe o uso de aprendizado de máquina para prever o 
consumo de cimento no país, além de investigar o desempenho dos modelos para 
identificar o mais eficiente para a tarefa. Foram testadas  regressão linear, 
redes \textit{feed forward}, redes recorrentes e redes bidirecionais, além disso, durante o trabalho foi realizado
um pré-processamento dos dados para garantir a granularidade necessária, tratar
de valores faltantes e assegurar que foram utilizados apenas dados do passado ao 
realizar a previsão. Por fim, nos experimentos realizados de forma a identificar o modelo com 
melhor desempenho, destaca-se o aumento na assertividade das previsões $-$ mensurada
a partir de métricas de erro absoluto e percentual $-$  à medida que 
se eleva a complexidade do modelo. Assim, o trabalho em questão investiga se é possível prever a demanda por 
cimento com assertividade ao utilizar modelos de aprendizado de máquina.}
 
\abstract{Cement is one of the main inputs of the civil construction industry, which is one of the biggest input of the Brazilian economy. The cement industry presents the demand of a way of forecasting cement consumption in Brazil, in order to target strategies and investments.
ments. This work, then, proposes the use of machine learning to forecasting cement consumption
in the country, in addition to investigating the performance of the models to identify the most efficient one for the task. They were
tested linear regression, feed forward networks, recurrent networks and bidirectional networks, in addition, during
the work was carried out a pre-processing of the data to guarantee the necessary granularity, treat
of missing values and ensure that only past data was used when making the prediction. 
Finally, in the experiments carried out in order to identify the model with the best performance, the
increase in the performance of forecasts $-$ measured from absolute and percentage error metrics $-$ to
as the complexity of the model increases. Thus, the work in question investigates whether it is possible to predict
the demand of cement with good performance when using machine learning models.
}
