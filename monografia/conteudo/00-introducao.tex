%!TeX root=../tese.tex
%("dica" para o editor de texto: este arquivo é parte de um documento maior)
% para saber mais: https://tex.stackexchange.com/q/78101

%% ------------------------------------------------------------------------- %%

% "\chapter" cria um capítulo com número e o coloca no sumário; "\chapter*"
% cria um capítulo sem número e não o coloca no sumário. A introdução não
% deve ser numerada, mas deve aparecer no sumário. Por conta disso, este
% modelo define o comando "\unnumberedchapter".

\unnumberedchapter{Introdução}
\label{cap:introducao}

\enlargethispage{.5\baselineskip}

O que prédios, pontes, hidrelétricas e aeroportos têm em comum? Todos são frutos da indútria da contrução civil, um importante componente do investimento brasileiro e, consequentemente, uma das grandes engrenagens responsáveis por movimentar a atividade econômica no Brasil. Em 2021, por exemplo, o Produto Interno Bruto (PIB) desse setor registrou alta de 9,7\%, enquanto o PIB do Brasil cresceu 4,6\%, assim, o setor da construção figura como importante impulsionador da economia do país. \cite{cbic-report} O cimento, nesse contexto, por ser um ingrediente central da argamassa e do concreto, caracteriza-se como um dos principais insumos da indústria.

\section*{Motivação}

Contudo, a falta de um modo bem fundamentado para prever a demanda de cimento é uma dor entre as empresas cimenteiras, uma vez que a construção de uma fábrica é custosa e demorada, além disso, aumentar a capacidade de produção de uma fábrica também é um processo custoso. 

\section*{Justificativa}

Dessa forma, um modelo que permitisse prever a demanda a nível de estados do Brasil poderia auxiliar gestores a tomar melhores decisões e a estruturar a estratégia de forma mais embasada, de modo a reduzir os riscos do setor. Além disso, poder-se-ia apoiar órgãos governamentais a direcionar ações para mitigar o impacto ambiental da fabricação desse produto.

\section*{Objetivos}


Este trabalho, então, propõe-se a  aplicar modelos de aprendizado de máquina para determinar qual é mais eficiente para prever a demanda por cimento nos estados do Brasil. Os modelos avaliados são: regressão linear, redes neurais  \textit{multi-layer perceptron} (MLP) e redes neurais recorrentes. Esses modelos recebem como entrada dados de crescimento econômico, de política monetária e fiscal, além de indicadores da construção civil e sociais de um determinado mês (ou de meses anteriores) e têm o intuito de estimar o consumo de cimento no mês seguinte.


%% ------------------------------------------------------------------------- %%
% \unnumberedsection{Considerações de estilo}
% \label{sec:consideracoes_preliminares}

% Normalmente, as citações não devem fazer parte da estrutura sintática da
% frase\footnote{E não se deve abusar das notas de rodapé.\index{Notas de rodapé}}.
% No entanto, usando referências em algum estilo autor-data (como o estilo
% plainnat do \LaTeX{}), é comum que o nome do autor faça parte da frase. Nesses
% casos, pode valer a pena mudar o formato da citação para não repetir o nome do
% autor; no \LaTeX{}, isso pode ser feito usando os comandos
% \textsf{\textbackslash{}citet}, \textsf{\textbackslash{}citep},
% \textsf{\textbackslash{}citeyear} etc. documentados no pacote
% natbib \citep{natbib}\index{natbib} (esses comandos são compatíveis com biblatex
% usando a opção \textsf{natbib=true}, ativada por padrão neste modelo). Em geral,
% portanto, as citações devem seguir estes exemplos:

% \footnotesize
% \begin{verbatim}
% Modos de citação:
% indesejável: [AF83] introduziu o algoritmo ótimo.
% indesejável: (Andrew e Foster, 1983) introduziram o algoritmo ótimo.
% certo: Andrew e Foster introduziram o algoritmo ótimo [AF83].
% certo: Andrew e Foster introduziram o algoritmo ótimo (Andrew e Foster, 1983).
% certo (\citet ou \citeyear): Andrew e Foster (1983) introduziram o algoritmo ótimo.
% \end{verbatim}
% \normalsize

% \enlargethispage{.5\baselineskip}

% O uso desnecessário de termos em língua estrangeira deve ser evitado. No entanto,
% quando isso for necessário, os termos devem aparecer \textit{em itálico}.
% \index{Língua estrangeira}
% % index permite acrescentar um item no indice remissivo

% Uma prática recomendável na escrita de textos é descrever as
% legendas\index{Legendas} das figuras e tabelas em forma auto-contida: as
% legendas devem ser razoavelmente completas, de modo que o leitor possa entender
% a figura sem ler o texto em que a figura ou tabela é citada.\index{Floats}

% \unnumberedsection{Ferramentas bibliográficas}

% Embora seja possível pesquisar por material acadêmico na Internet usando
% sistemas de busca ``comuns'', existem ferramentas dedicadas, como o
% \textsf{Google Scholar}\index{Google Scholar} (\url{scholar.google.com}).
% O \textsf{Web of Science}\index{Web of Science}
% (\url{webofscience.com}) e o \textsf{Scopus}\index{Scopus} (\url{scopus.com})
% oferecem recursos sofisticados e limitam a busca a periódicos com boa
% reputação acadêmica. Essas duas plataformas não são gratuitas, mas os alunos
% da USP têm acesso a elas através da instituição. Algumas editoras, como a
% ACM (\url{dl.acm.org}) e a IEEE (\url{ieeexplore.ieee.org}), também têm
% sistemas de busca bibliográfica. Todas essas ferramentas são capazes de
% exportar os dados para o formato .bib, usado pelo \LaTeX{} (no Google
% Scholar, é preciso ativar a opção correspondente nas preferências). O sítio
% \url{liinwww.ira.uka.de/bibliography} também permite buscar e baixar
% referências bibliográficas relevantes para a área de computação.

% Lamentavelmente, ainda não existe um mecanismo de verificação ou validação
% das informações nessas plataformas. Portanto, é fortemente sugerido validar
% todas as informações de tal forma que as entradas bib estejam corretas.
% De qualquer modo, tome muito cuidado na padronização das referências
% bibliográficas: ou considere TODOS os nomes dos autores por extenso, ou
% TODOS os nomes dos autores abreviados. Evite misturas inapropriadas.\looseness=-1

% Apenas uma parte dos artigos acadêmicos de interesse está disponível livremente
% na Internet; os demais são restritos a assinantes. A CAPES assina um grande
% volume de publicações e disponibiliza o acesso a elas para diversas universidades
% brasileiras, entre elas a USP, através do seu portal de periódicos
% (\url{periodicos.capes.gov.br}). Existe uma extensão para os navegadores
% Chrome e Firefox (\url{www.infis.ufu.br/capes-periodicos}) que facilita o uso
% cotidiano do portal.

% Para manter um banco de dados organizado sobre artigos e outras fontes bibliográficas
% relevantes para sua pesquisa, é altamente recomendável que você use uma ferramenta
% como Zotero~(\url{zotero.org})\index{Zotero} ou
% Mendeley~(\url{mendeley.com})\index{Mendeley}. Ambas podem exportar seus dados no
% formato .bib, compatível com \LaTeX{}.
