%!TeX root=../tese.tex
%("dica" para o editor de texto: este arquivo é parte de um documento maior)
% para saber mais: https://tex.stackexchange.com/q/78101

%% ------------------------------------------------------------------------- %%

% "\chapter" cria um capítulo com número e o coloca no sumário; "\chapter*"
% cria um capítulo sem número e não o coloca no sumário. A introdução não
% deve ser numerada, mas deve aparecer no sumário. Por conta disso, este
% modelo define o comando "\unnumberedchapter".

\unnumberedchapter{Introdução}
\label{cap:introducao}

\enlargethispage{.5\baselineskip}

O que prédios, pontes, hidrelétricas e aeroportos têm em 
comum? Todos são frutos da indútria da contrução civil, 
um importante componente do investimento brasileiro e, 
consequentemente, uma das grandes engrenagens responsáveis 
por movimentar a atividade econômica no Brasil. Em 2021, 
por exemplo, segundo \cite{cbic-report}, o Produto Interno Bruto (PIB) desse setor 
destacou-se com alta de 9,7\% enquanto o PIB do Brasil cresceu 
4,6\% e o PIB do Agronegócio registrou queda de 0,2\%. Dessa 
forma, o setor da construção figura como importante 
impulsionador da economia do país, capaz de gerar renda e 
empregos. 

Nesse contexto, o cimento\footnote{O material conhecido na contrução 
civil como "cimento" é denominado mundialmente como cimento de 
portland. Em 1824, construções com pedra de Portland eram 
comuns na Inglaterra, por isso, o inventor do cimento, Joseph
Aspdin, ao notar que sua invenção tinha aspecto similar ao 
material tão difundido na época, optou por registrar a patente
como "cimento de portland".} é uma das principais matérias
primas da indústria da construção civil. Esse material é 
um pó com propriedades aglomeramentes que endurece quando é 
submetido à água, após endurecer, contudo, não é 
mais decomposto, mesmo em contato com a água. Além disso,
ao ser misturado com areia, pedra 
britada, pó-de-pedra e cal resulta na argamassa e no 
concreto utilizados em construções. (\cite{boletim-cimento})

\section*{Motivação}

Contudo, a falta de um modo bem fundamentado para prever o 
consumo de cimento nos estados do Brasil é 
uma demanda entre as empresas cimenteiras. Uma nova fábrica 
representa para a empresa um alto investimento a longo prazo,
uma vez que a contrução de uma fábrica representa um alto investimento
financeiro, além de tempo, por demorar anos para finalizar as obras, 
a exemplo da fábrica construída pela Votorantim Cimentos no Complexo 
Industrial e Portuário do Pecém, no Ceará, cuja construção 
demorou 3 anos e custou cerca de R\$ 200 milhões, conforme 
noticiado no portal do Instituto Brasileiro de Mineração 
(\cite{fabrica-noticia}).
Além disso, o cimento é um produto que não suporta longas 
estadias em estoque, visto que a norma brasileira recomenda
o uso em até 90 dias após a fabricação, é importante, então, que 
a produção do material esteja fortemente alinhada ao 
consumo. (\cite{abnt})

Assim, um modelo que permitisse prever a demanda a nível de 
estados do Brasil poderia auxiliar gestores a tomar melhores 
decisões e a estruturar a estratégia de forma mais embasada, de 
modo a reduzir os riscos do setor. Além disso, poder-se-ia apoiar órgãos 
governamentais a direcionar ações para mitigar o impacto ambiental da
fabricação desse produto, pontuado como um grande emissor de 
gases de efeito-estufa em escala mundial.(\cite{meio-ambiente})

%valeu a pena a grana da votorantin?

\section*{Objetivos}

Este trabalho, então, propõe-se a  aplicar modelos de aprendizado 
de máquina para determinar qual é mais eficiente para prever a 
demanda por cimento nos estados do Brasil. Os algoritmos avaliados em ordem 
crescente de complexidade e robustez são: 
regressão linear, redes neurais  \textit{multi-layer perceptron} (MLP) 
e redes neurais recorrentes.


%% ------------------------------------------------------------------------- %%
% \unnumberedsection{Considerações de estilo}
% \label{sec:consideracoes_preliminares}

% Normalmente, as citações não devem fazer parte da estrutura sintática da
% frase\footnote{E não se deve abusar das notas de rodapé.\index{Notas de rodapé}}.
% No entanto, usando referências em algum estilo autor-data (como o estilo
% plainnat do \LaTeX{}), é comum que o nome do autor faça parte da frase. Nesses
% casos, pode valer a pena mudar o formato da citação para não repetir o nome do
% autor; no \LaTeX{}, isso pode ser feito usando os comandos
% \textsf{\textbackslash{}citet}, \textsf{\textbackslash{}citep},
% \textsf{\textbackslash{}citeyear} etc. documentados no pacote
% natbib \citep{natbib}\index{natbib} (esses comandos são compatíveis com biblatex
% usando a opção \textsf{natbib=true}, ativada por padrão neste modelo). Em geral,
% portanto, as citações devem seguir estes exemplos:

% \footnotesize
% \begin{verbatim}
% Modos de citação:
% indesejável: [AF83] introduziu o algoritmo ótimo.
% indesejável: (Andrew e Foster, 1983) introduziram o algoritmo ótimo.
% certo: Andrew e Foster introduziram o algoritmo ótimo [AF83].
% certo: Andrew e Foster introduziram o algoritmo ótimo (Andrew e Foster, 1983).
% certo (\citet ou \citeyear): Andrew e Foster (1983) introduziram o algoritmo ótimo.
% \end{verbatim}
% \normalsize

% \enlargethispage{.5\baselineskip}

% O uso desnecessário de termos em língua estrangeira deve ser evitado. No entanto,
% quando isso for necessário, os termos devem aparecer \textit{em itálico}.
% \index{Língua estrangeira}
% % index permite acrescentar um item no indice remissivo

% Uma prática recomendável na escrita de textos é descrever as
% legendas\index{Legendas} das figuras e tabelas em forma auto-contida: as
% legendas devem ser razoavelmente completas, de modo que o leitor possa entender
% a figura sem ler o texto em que a figura ou tabela é citada.\index{Floats}

% \unnumberedsection{Ferramentas bibliográficas}

% Embora seja possível pesquisar por material acadêmico na Internet usando
% sistemas de busca ``comuns'', existem ferramentas dedicadas, como o
% \textsf{Google Scholar}\index{Google Scholar} (\url{scholar.google.com}).
% O \textsf{Web of Science}\index{Web of Science}
% (\url{webofscience.com}) e o \textsf{Scopus}\index{Scopus} (\url{scopus.com})
% oferecem recursos sofisticados e limitam a busca a periódicos com boa
% reputação acadêmica. Essas duas plataformas não são gratuitas, mas os alunos
% da USP têm acesso a elas através da instituição. Algumas editoras, como a
% ACM (\url{dl.acm.org}) e a IEEE (\url{ieeexplore.ieee.org}), também têm
% sistemas de busca bibliográfica. Todas essas ferramentas são capazes de
% exportar os dados para o formato .bib, usado pelo \LaTeX{} (no Google
% Scholar, é preciso ativar a opção correspondente nas preferências). O sítio
% \url{liinwww.ira.uka.de/bibliography} também permite buscar e baixar
% referências bibliográficas relevantes para a área de computação.

% Lamentavelmente, ainda não existe um mecanismo de verificação ou validação
% das informações nessas plataformas. Portanto, é fortemente sugerido validar
% todas as informações de tal forma que as entradas bib estejam corretas.
% De qualquer modo, tome muito cuidado na padronização das referências
% bibliográficas: ou considere TODOS os nomes dos autores por extenso, ou
% TODOS os nomes dos autores abreviados. Evite misturas inapropriadas.\looseness=-1

% Apenas uma parte dos artigos acadêmicos de interesse está disponível livremente
% na Internet; os demais são restritos a assinantes. A CAPES assina um grande
% volume de publicações e disponibiliza o acesso a elas para diversas universidades
% brasileiras, entre elas a USP, através do seu portal de periódicos
% (\url{periodicos.capes.gov.br}). Existe uma extensão para os navegadores
% Chrome e Firefox (\url{www.infis.ufu.br/capes-periodicos}) que facilita o uso
% cotidiano do portal.

% Para manter um banco de dados organizado sobre artigos e outras fontes bibliográficas
% relevantes para sua pesquisa, é altamente recomendável que você use uma ferramenta
% como Zotero~(\url{zotero.org})\index{Zotero} ou
% Mendeley~(\url{mendeley.com})\index{Mendeley}. Ambas podem exportar seus dados no
% formato .bib, compatível com \LaTeX{}.
